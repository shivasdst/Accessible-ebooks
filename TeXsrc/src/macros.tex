\usepackage{polyglossia}
\usepackage{fontspec,graphicx}
\usepackage{setspace}
\usepackage{enumitem}
\usepackage{array}
\usepackage{float}
\usepackage{color}

\floatstyle{plain}
\floatplacement{Figure}{tbp}

\usepackage[papersize={176mm,250mm},textwidth=145mm, textheight=200mm,
headheight=6mm,headsep=5mm,topmargin=20mm,botmargin=19mm,
leftmargin=17mm,rightmargin=17mm,footskip=7mm,cropmarks]{zwpagelayout} % 1/4 crown size


\newenvironment{descriptionSimple}[1]%
  {\begin{list}{}{\renewcommand{\makelabel}[1]{\textbf{##1:}\hfil}%
    \settowidth{\labelwidth}{\textbf{#1:}}%
    \setlength{\itemsep}{0ex}%
    \setlength{\topsep}{-1ex}%
    \setlength{\parsep}{0ex}%
    \setlength{\partopsep}{0ex}%
    \setlength{\leftmargin}{\labelwidth}\addtolength{\leftmargin}{\labelsep}}}%
  {\end{list}}

\setdefaultlanguage{hindi}
%\setmainfont[Script=Devanagari,Ligatures=TeX,AutoFakeBold=3.5,AutoFakeSlant,WordSpace=1.2]{ApuDevNewUnicode}
\setmainfont[Script=Devanagari,Ligatures=TeX,AutoFakeSlant,WordSpace=1.2]{ApuDevNewUnicode}
\XeTeXgenerateactualtext 1
\newfontscript{Devanagari}{deva,dev2}
\def\paragraphTitle#1{\medbreak\noindent{\bfseries \color{red}{#1}}\\}
\font\eng = cmr10 at 11pt
% Define the \eng command as a localized version of \latinfont

\linespread{1.1}
\parindent=0pt
\setlength{\parskip}{\baselineskip}

\renewcommand{\arraystretch}{1.4}
\def\captionshindi{%
     \def\abstractname{सारांश}%
     \def\appendixname{परिशिष्ट}%
     \def\bibname{संदर ग्रन्थ}% (?)
     \def\ccname{}%
     \def\chaptername{अध्याय}%
     \def\contentsname{विषय सूची}%
     \def\enclname{}%
     \def\figurename{चित्र}% रेखाचित्र
     \def\headpagename{पृषठ}%
     \def\headtoname{}%
     \def\indexname{सूची}%
     %              सूचक
     %              अनुक्रमणिका
     %              अनुक्रमणि
     \def\listfigurename{चित्रों की सूची}%
     \def\listtablename{तालिकाओं की सूची}%
     \def\pagename{पृषठ}%
     \def\partname{खणड}%
     \def\prefacename{प्रस्तावना}% प्राक्कथन
     \def\refname{हवाले}%
     \def\tablename{टेबल}%
     \def\seename{देखिए}%
     \def\alsoname{और देखिए}%
	 \def\alsoseename{और देखिए}%
}

\makeatletter
\def\ps@plain{\let\@mkboth\@gobbletwo
     \let\@oddhead\@empty\def\@oddfoot{\reset@font\hfil\eng\thepage
     \hfil}\let\@evenhead\@empty\let\@evenfoot\@oddfoot}
\def\@makechapterhead#1{%
  \vspace*{50\p@}%
  {\parindent \z@ \raggedright \normalfont
    \ifnum \c@secnumdepth >\m@ne
      \if@mainmatter
        {\huge\bfseries \color{red}{\@chapapp\space \thechapter}}
        \par\nobreak
        \vskip 20\p@
      \fi
    \fi
    \interlinepenalty\@M
    {\Huge \bfseries \color{red}{#1}}\par\nobreak
    \vskip 40\p@
  }}
\renewcommand\section{\@startsection {section}{1}{\z@}%
                                   {-0.5ex \@plus -0.5ex \@minus -.1ex}%
                                   {0.5ex \@plus.2ex}%
                                   {\color{red}\normalfont\Large\bfseries}}
\renewcommand\labelitemi{{\fontfamily{txr}\selectfont\textbullet}}
\makeatother                                   
                                   
\pagestyle{plain}
